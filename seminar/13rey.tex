\subsection{Reyleigh's Stability Equation}
To deal with the simple problem in hydrodynamic stability, adopt the
small amplitude linear approximation to define the velocity,
pressure, and assume the fluid is incompressible:
\begin{subequations}\label{kh:ln}
\begin{align}
    \mathbf{u}(\mathbf{x},t)&=\mathbf{U}(z)+\mathbf{u}'(\mathbf{x},t)\label{kh:lnu}\\
    p(\mathbf{x},t)&=P(z)+p'(\mathbf{x},t)\label{kh:lnp}\\
    \rho&=\rho_0\quad\text{everywhere}\label{kh:lnrho}
\end{align}
\end{subequations}
where $\mathbf{u}(\mathbf{x},t)$ and $p(\mathbf{x},t)$ is the
velocity and pressure at different location and time,
$\mathbf{U}(z)$ is the background velocity , $P(z)$ is the
background pressure. The background fields only depend on $z$.
$\mathbf{u}'(\mathbf{x},t)$ and $p'(\mathbf{x},t)$ are the perturbed
velocity and pressure fields respectively.
\newslide
Then write the Navier-Stokes equation explicitly
\begin{equation}\label{kh:nav}
\frac{\partial}{\partial
t}(\mathbf{U}(z)+\mathbf{u}')+(\mathbf{U}(z)+\mathbf{u}')
\cdot\nabla(\mathbf{U}(z)+\mathbf{u}')=-\frac{\nabla(P(z)+p')}{\rho_0}-g\mathbf{k}
\end{equation}
Since $g=-(1/\rho_0)(\partial P/\partial z)$, the background
pressure term and the gravitational force term canceled each other.
\newslide
\eqref{kh:nav} can be linearized by neglecting products of the small
perturbed quantities (denotes by primes).
\begin{equation}\label{kh:lin}
(\frac{\partial}{\partial t}+U\frac{\partial}{\partial
x})\mathbf{u}'+\frac{dU}{dz}w'\mathbf{i}=-\frac{\nabla p'}{\rho_0}
\end{equation}
where $\mathbf{u}' = (u',v',w')$.
\newslide
The equation has coefficients indepedent of $x$, $y$, $t$ but not
$z$.

Separate the variables by taking independent normal modes of the
form
\begin{equation}\label{kh:modes}
    \mathbf{u}'(\mathbf{x},t)=\hat{\mathbf{u}}(z)e^{i(\alpha x+\beta y-\alpha c t)},\qquad
    p'(\mathbf{x},t)=\hat{p}(z)e^{i(\alpha x+\beta y-\alpha c t)}
\end{equation}
\newslide
Together with \eqref{mass:cont3} the equation of continuity
$\nabla\cdot\mathbf{u}=0$, \eqref{kh:lin} gives
\begin{subequations}\label{kh:1}
\begin{align}
i\alpha(U-c)\hat{u}+\frac{dU}{dz}\hat{w}+\frac{i\alpha}{\rho_0}\hat{p}&=0\label{kh:1-1}\\
i\alpha(U-c)\hat{v}+\frac{i\beta}{\rho_0}\hat{p}&=0\label{kh:1-2}\\
i\alpha(U-c)\hat{w}+\frac{1}{\rho_0}\frac{d\hat{p}}{dz}&=0\label{kh:1-3}\\
i\alpha\hat{u}+i\beta\hat{v}+\frac{d\hat{w}}{dz}&=0\label{kh:1-4}
\end{align}
\end{subequations}
%-----------------------------------------------------------
\newslide
Apply the Squire's transformation
\begin{equation}\label{kh:squire}
    \tilde{\alpha}=(\alpha^2+\beta^2)^{1/2},\qquad
\tilde{u}=(\alpha\hat{u}+\beta\hat{v})/\tilde{\alpha},\qquad
\tilde{p}=\tilde{\alpha}\hat{p}/\alpha
\end{equation}
Then multiply \eqref{kh:1-1} by $\alpha$ and \eqref{kh:1-2} by
$\beta$, take the sum and divide by $\alpha$ to get \eqref{kh:2-1}.
\eqref{kh:1-3} and \eqref{kh:1-4} can be transformed in a similar
way.
\begin{subequations}\label{kh:2}
\begin{align}
i\tilde{\alpha}(U-c)\tilde{u}+\frac{dU}{dz}\hat{w}+\frac{i\tilde{\alpha}}{\rho_0}\tilde{p}&=0\label{kh:2-1}\\
i\tilde{\alpha}(U-c)\hat{w}+\frac{1}{\rho_0}\frac{d\tilde{p}}{dz}&=0\label{kh:2-2}\\
i\tilde{\alpha}\tilde{u}+\frac{d\hat{w}}{dz}&=0\label{kh:2-3}
\end{align}
\end{subequations}
\newslide
The three-dimensional mode in \eqref{kh:1} is essentially reduced to
a two-dimensional mode of wave number vector
$\boldsymbol{\alpha}=\alpha\mathbf{i}+\beta\mathbf{j}$, for
$\tilde{\alpha}=\lvert\boldsymbol{\alpha}\rvert$ and
$\tilde{u}=\hat{\boldsymbol{\alpha}}\cdot\hat{\mathbf{u}}$, where
$\hat{\boldsymbol{\alpha}}=\boldsymbol{\alpha}/\tilde{\alpha}$. This
represents a wave traveling in the direction of
$\boldsymbol{\alpha}$. \eqref{kh:1} is the same as \eqref{kh:2} when
$\beta=\hat{v}=0$.
\newslide
Introduce a perturbed stream function $\psi'$ into the linearized
equations \eqref{kh:2}. Define that
\begin{equation}\label{kh:str1}
    u'=\frac{\partial\psi'}{\partial z},\qquad v'=0,\qquad w'=-\frac{\partial\psi'}{\partial x}
\end{equation}
and take independent normal modes of the form
\begin{equation}\label{kh:modes2}
    \psi'(x,z,t)=\phi(z)e^{i\alpha(x-ct)}
\end{equation}
\newslide
It follows that
\begin{equation}\label{kh:str2}
    \hat{u}=\frac{d\phi}{dz},\qquad \hat{w}=-i\alpha\phi
\end{equation}
which will automatically satisfy the equation of continuity
\eqref{kh:1-4}.
\newslide
From \eqref{kh:1-1},
\begin{equation}\label{kh:3-1}
    \frac{\hat{p}}{\rho_0}=\frac{dU}{dz}\phi-(U-c)\frac{d\phi}{dz}
\end{equation}
From \eqref{kh:1-3},
\begin{equation}\label{kh:3-2}
    (U-c)\alpha^2\phi+\frac{1}{\rho_0}\frac{d\hat{p}}{dz}=0
\end{equation}
Differentiate \eqref{kh:3-1} with respect to $z$ and substitute into
\eqref{kh:3-2}, we get the Rayleigh's stability equation:
\begin{equation}\label{kh:ray}
    \boxed{(U-c)(\frac{d^2\phi}{dz^2}-\alpha^2\phi)-\frac{d^2U}{dz^2}\phi=0}
\end{equation}
\newslide
For a piecewise linear velocity profile, the term $d^2U/dz^2$
vanishes, the Rayleigh's stability equation reduces to a second
order linear differential equation
\begin{equation}\label{kh:ray2}
    \frac{d^2\phi}{dz^2}-\alpha^2\phi=0
\end{equation}

The general solution is in the form $\phi = Ae^{-\alpha
z}+Be^{\alpha z}$.
