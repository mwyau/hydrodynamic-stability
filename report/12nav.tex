\section{Navier-Stokes Equation}
The fundamental equation in fluid mechanics, Navier-Stokes equation,
is derived from conservation of momentum.
\begin{equation*}
    \rho\frac{D u_i}{D t} = \rho f_i + \frac{\partial}{\partial
    x_j}\sigma_{ij}
\end{equation*}
The stress tensor is symmetric and isotropic in a static fluid.
Define the static stress as
\begin{equation}\label{nav1}
    \sigma_{ij}=-p\delta_{ij}
\end{equation}
In general, for a moving fluid we can expand $\sigma_{ij}$ into
\begin{equation}\label{nav2}
    \sigma_{ij}=-p\delta_{ij}+d_{ij}
\end{equation}
where $d_{ij}$ is due to the fluid motion alone, called the
deviatoric stress tensor. $d_{ij}$ cannot be only related to the
coordinate $x_j$ since it is independent of reference
frame\footnote{If $d_{ij}$ only depends on $x_j$, upon changing to a
co-moving frame with the fluid, $d_{ij}$ will be zero. But stress
should be invariant from coordinate transformation.}. Use the
Newtonian hypothesis that it is a linear function of $\partial
u_i/\partial x_j$, we can express the relationship with a 4-th rank
tensor:
\begin{equation}\label{nav3}
    d_{ij}=A_{ijkl}\frac{\partial u_k}{\partial x_l}
\end{equation}
$\partial u_i/\partial x_j$ can be decomposed into a symmetric and
an antisymmetic part:
\begin{align*}
    \frac{\partial u_i}{\partial x_j}&=
    \frac{1}{2}\Bigl(\frac{\partial u_i}{\partial x_j}+\frac{\partial u_j}{\partial
    x_i}\Bigr)
    +\frac{1}{2}\Bigl(\frac{\partial u_i}{\partial
    x_j}-\frac{\partial u_j}{\partial
    x_i}\Bigr)\\
    &=\frac{1}{2}\Bigl(\frac{\partial u_i}{\partial x_j}+\frac{\partial u_j}{\partial
    x_i}\Bigr)
    -\frac{1}{2}\epsilon_{ijk}\,\omega_k
\end{align*}
where $\mathbf{\omega}=\nabla\times\mathbf{u}$ is the fluid
vorticity. Define $e_{ij}=\dfrac{1}{2}\Bigl(\dfrac{\partial
u_i}{\partial x_j}+\dfrac{\partial u_j}{\partial
    x_i}\Bigr)$ and
$\zeta_{ij}=-\frac{1}{2}\epsilon_{ijk}\,\omega_k$,
\begin{equation}\label{nav4}
    d_{ij}=A_{ijkl}\frac{\partial u_i}{\partial
    x_j}=A_{ijkl}(e_{kl}+\zeta_{kl})
\end{equation}
Assume the fluid is isotropic in motion, that $d_{ij}$ is
independent of the orientation of the fluid. Then $A_{ijkl}$ must
also be isotropic. From Jeffreys \cite[p.70]{Tensor}, the general
4th order isotropic tensor has the form
\begin{equation}\label{nav5}
    A_{ijkl}=\mu\delta_{ik}\delta_{jl}+\mu'\delta_{il}\delta_{jk}+\mu''\delta_{ij}\delta_{kl}
\end{equation}
Since the stress is symmetric,
\begin{align}
    \sigma_{ij}&=\sigma_{ji}\notag\\
    \mu&=\mu'\notag\\
    A_{ijkl}&=2\mu\delta_{ik}\delta_{jl}+\mu''\delta_{ij}\delta_{kl}
\end{align}
We can see that $A_{ijkl}$ is also symmetric in $l$ and $k$. Then
\begin{align}
    A_{ijkl}\zeta_{ij}&=-\frac{1}{2}A_{ijkl}\,\epsilon_{ijk}\,\omega_k=0\notag\\
    d_{ij}&=A_{ijkl}e_{kl}\notag\\
    &=(2\mu\delta_{ik}\delta_{jl}+\mu''\delta_{ij}\delta_{kl})e_{kl}\notag\\
    &=2\mu e_{ij}+\mu''e_{ll}\delta_{ij}\notag\\
    &=2\mu e_{ij}+\mu''\Delta\delta_{ij}
\end{align}
where $\Delta=\nabla\cdot\mathbf{u}$. Also by definition of
$d_{ij}$,
\begin{align}
    \text{tr}(d_{ij})&\equiv 0 = d_{ii}\notag\\
    d_{ii}&=2\mu e_{ii}+3\mu''\Delta=0\notag\\
    \text{if $e_{ii}\neq 0$},\qquad \mu''&=-\frac{2}{3}\mu
\end{align}
Therefore
\begin{align}
    d_{ij}=2\mu(e_{ij}-\frac{1}{3}\Delta\delta_{ij})\notag\\
    \sigma_{ij}=-p\delta_{ij}+2\mu(e_{ij}-\frac{1}{3}\Delta\delta_{ij})\label{nav6}
\end{align}
$\mu$ is called the molecular viscosity. Finally we get the
Navier-Stokes equation by substituting \eqref{nav6} into the
conservation of momentum equation:
\begin{equation}\label{nav:eq}
    \boxed{\rho\frac{D u_i}{D t} = \rho f_i - \frac{\partial p}{\partial
    x_i}+\frac{\partial}{\partial x_j}\Bigl(2\mu(e_{ij}-\frac{1}{3}\Delta\delta_{ij})\Bigr)}
\end{equation}
If $\Delta=\nabla\cdot\mathbf{u}=0$, with $\mu$ is constant,
\begin{equation}\label{nav:eq2}
    \rho\frac{D u_i}{D t} = \rho f_i - \frac{\partial p}{\partial
    x_i}+\mu\nabla^2 u_i
\end{equation}
Or in vector form:
\begin{equation}\label{nav:eq3}
    \rho\frac{D \mathbf{u}}{D t} = \rho \mathbf{f} - \nabla p + \mu\nabla^2 \mathbf{u}
\end{equation}
